\documentclass[a4paper,12pt]{article}

\usepackage[utf8]{inputenc}
\usepackage{polski}
\usepackage{a4wide}

\usepackage[colorlinks=true,linkcolor=blue,urlcolor=blue]{hyperref}

\usepackage[pdftex]{graphicx}



\hypersetup{
        unicode = true,
        pdfauthor = {Szymon Piątek},
}

\title{Instalacja oprogramowania}
\author{Szymon Piątek}

\hyphenation{IaaS Rack-Space}

\begin{document}
\maketitle

\section{Aktualizacja systemu w~laboratorium}
\subsection{Praca na maszynie wirtualnej}
Praca wspólna: Jacek Kołodziejski, Szymon Piątek, Piotr Wasilewski

\begin{enumerate}
        
        \item Ustawiono MAC na VM~ oraz opcję bridge
        \item Wyłączono komputer o~danym MACu
        \item Ustawiono Virtualizację i~VT--intel w~BIOSie komputera z~VM
        \item Ustawiono rozmiar dysku VM~na większy niż 16GB
        \item Po pojawieniu się ikonki Ubuntu:~uruchomiono instalator bez modułów związanych z~zaawansowanymi przerwaniami
        \footnote{Pod maszyną wirtualną moduły te powodują kernel panic}
        \begin{verbatim}ux noapic nolapic\end{verbatim}
        \item Gdy nie uda się ściąganie: przejść do następnego kroku
        \item Usunięto instalację i konfigurację Virtualbox - problemy z instalatorem
        \item Przygotowanie instalacji Ubuntu 10.10: Stworzono katalogi preseed2 i~tftp/rso
        \item Dodano linijki z nową instalacją (rso) w~tftp
        \item Uruchamianie instalacji Ubuntu 10.10
        \begin{verbatim} rso noapic nolapic \end{verbatim}
        \item W~post-install dodano instalację ethtool przy okazji wakeonlane, zamienieno librrd2 na librrd4
        \item Dodano plik nsswitch.conf do tftp, jest on ściągany w~post\-install po zainstalowaniu systemu. Umożliwia on dostęp do ldap
        \item Do pliku preseed.conf dodano potwierdzenie partycjonowania (usunięcie pytania z~instalatora)

\end{enumerate}

\bibliography{../bibliografia}
\bibliographystyle{../plplainurl}

\end{document}
                       
