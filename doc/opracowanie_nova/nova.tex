\documentclass[a4paper,10pt]{article}

\usepackage[utf8]{inputenc}
\usepackage{polski}
\usepackage{microtype}

\usepackage{amsmath}
\usepackage{listings}

\title{OpenStack Compute (Nova)\\[1em]
\large \sc \textls{Opracowanie}}
\author{Daniel Kłobuszewski}

\begin{document}
\maketitle

\section{Wstęp}
Obecnie często mamy do czynienia z sytuacją, gdzie serwer, świadczący pewną usługę, jest w pełni obciążony tylko okresowo lub wcale. Przeniesienie takiego serwera na maszynę wirtualną oraz zgrupowanie go z innymi pozwala rozwiązać ten problem. Możemy dzięki temu uruchomić dużą ilość maszyn wirtualnych na mniejszej ilości maszyn fizycznych.

Projekt Nova (czyli OpenStack Compute) jest implementacją takiej infrastruktury. Poniższy dokument ma na celu określenie funkcjonalności oraz możliwości konfiguracji projektu OpenStack Compute.

\section{Możliwości}
Nova realizuje model IaaS -- Infrastructure as a Service (infrastruktura jako usługa). W modelu tym dostawca posiada infrastrukturę, w ramach której klient płaci za używaną przez siebie moc obliczeniową oraz przestrzeń dyskową. W przypadku obliczeń w chmurze, klient płaci za faktyczną ilość użytych zasobów\ref{yt:prezentacja_od_jana}, nie za wartość nominalną, jak ma to miejsce w przypadku fizycznych serwerów. 

Projekt ten dostarcza nam szereg funkcjonalności, które pomagają w realizacji obliczeń w chmurze. Mamy możliwość konfiguracji wielu parametrów -- od uprawnień użytkowników zaczynając, na topologii fizycznej tegoż cloud'a kończąc.

\section{Uprawnienia}

Wprowadzony jest podział użytkowników na role oraz projekty. W oparciu o nie określane są uprawnienia.

Mamy pięć ról do przydzielania użytkownikom\ref{nova_doc}:

\begin{itemize}
\item Cloud Administrator -- pełen zestaw uprawnień
\item IT Security -- umożliwia izolowanie instancji (działających maszyn wirtualnych)
\item Project Manager -- możliwość dodawania użytkowników do projektu, a także włączania/wyłączania instancji maszyn wirtualnych w projekcie
\item Network Administrator -- możliwość przypisywania publicznych adresów IP i modyfikacji ustawień firewalla
\item Developer -- domyślna, standardowa rola
\end{itemize}

Dodatkowo, OpenStack Compute określa pojęcie projektu. Projekty grupują zasoby i użytkowników, którzy mają do nich dostęp.

Istnieje szereg parametrów konfigurowalnych per projekt\ref{nova_doc}:

\begin{itemize}
\item Ilość wirtualnych dysków twardych w projekcie
\item Sumaryczny rozmiar dysków wirtualnych
\item Dopuszczalna ilość działających jednocześnie maszyn wirtualnych
\item Ilość alokowanych rdzeni procesora
\item Publiczne adresy IP
\end{itemize}

\section{Komponenty}

Istnieje kilka rodzajów komponentów, z których składa się cloud OpenStack Compute\ref{nova_doc}.
%TODO:    prawie -> napisać który nie jest
Aktualnie prawie każdy z nich jest łatwo skalowalny horyzontalnie, tzn. można w razie potrzeby dołożyć więcej elementów danego typu. Każdy z nich może być uruchamiany na osobnej maszynie, nic nie stoi też na przeszkodzie, aby je wszystkie uruchomić na jednej maszynie.

\subsection{Cloud controller}
Reprezentuje stan chmury? Ma jakąś bazę danych. Ma kolejkę? Jest jeden? %TODO: ogarnąć wtf

\subsection{API server}
Jedyny serwis dostępny z zewnątrz - przez niego przechodzą wszystkie żądania uruchomienia/wstrzymania maszyn wirtualnych. Komunikacja z nim odbywa się poprzez protokół HTTP. 

\subsection{Compute controller}
Serwer roboczy, na którym uruchamiane są maszyny wirtualne.

\subsection{Auth manager}
Autoryzuje oraz uwierzytelnia użytkowników. Przechowuje informacje o istniejących projektach i rolach.

\subsection{Volume controller}
Zarządza wirtualnymi dyskami. Każdy taki dysk można podłączyć do dowolnej działającej instancji maszyny wirtualnej. Serwis ten korzysta z iSCSI oraz Linuxowych LVM (Logical Volume Manager).

Jeden taki wirtualny dysk może być podłączony wyłącznie do jednej działającej instancji w danej chwili.

\subsection{Network controller}
Zajmuje się konfiguracją sieci. Może działać w trzech trybach:
\begin{itemize}
\item Flat -- 'Płaska' konfiguracja, ze statyczną konfiguracją adresów IP
\item FlatDHCP -- Jak wyżej, ale z serwerem DHCP.
\item VLAN -- Każdy projekt ma swój VLAN, w którym działa DHCP. Konfiguracja domyślna.
\end{itemize}

\subsection{Scheduler}
Komponent decydujący, który {\it compute controller} powinien obsłużyć maszynę wirtualną.

\subsection{Object storage}
Opcjonalny komponent, udostępniający obrazy maszyn wirtualnych. Zamiast niego, Compute może wykorzystywać serwer Glance.

\section{Wirtualizacja}

Nova pozwala na wykorzystanie poniższych typów wirtualizacji\ref{nova_doc}:

\begin{itemize}
\item{KVM}
\item{UML}
\item{XEN}
\item{Hyper-V}
\item{QEMU}
\end{itemize}

Obrazy maszyn wirtualnych mogą być przechowywane przy użyciu Glance (OpenStack Imaging Service) lub nova-objectstore.

%TODO: dodać bibliografię
\bibliography{bibref}{}
\bibliographystyle{plain}
	
\end{document}
