\documentclass[a4paper,12pt]{article}

\usepackage[utf8]{inputenc}
\usepackage{polski}
\usepackage{a4wide}

\usepackage[colorlinks=true,linkcolor=blue,urlcolor=blue]{hyperref}

\usepackage[pdftex]{graphicx}



\hypersetup{
	unicode = true,
	pdfauthor = {Jan Stępień},
}

\title{Analiza atechniczna}
\author{Jan Stępień}

\hyphenation{IaaS Rack-Space}

\begin{document}
	\maketitle

	\section{Wstęp}

	Projekt OpenStack to zestaw narzędzi umożliwiających utworzenie platformy
	typu IaaS, będącej podstawową warstwą zyskującej popularność architektury
	w~chmurze. W~jego skład wchodzą m.in. komponenty ułatwiające nadzorowanie
	pracy maszyn wirtualnych oraz umożliwiające redundancyjne przechowywanie
	plików i~dostęp do nich. Projekt ma charakter open source a~jego kod jest
	publicznie dostępny.

	Celem niniejszego dokumentu jest przedstawienie wyników subiektywnej,
	nietechnicznej analizy projektu OpenStack. Motywacją do jej przygotowania
	jest przekonanie autora, że takie cechy jak bogata funkcjonalność,
	kompleksowy garnitur testów i~bogata dokumentacja nie są warunkami
	wystarczającymi do zaufania danemu rozwiązaniu. W~projektach o~otwartym
	kodzie źródłowym istotną rolę odgrywa wiele innych czynników, takich jak
	choćby przezroczysty proces wytwórczy, efektywność kanałów komunikacji
	między użytkownikami i~twórcami oraz aktywna społeczność wspierająca prace
	nad rozwojem oprogramowania.

	\section{Historia}

	W~niniejszej sekcji pokrótce nakreślona zostanie historia projektu
	OpenStack. Przedstawione poniżej informacje w~głównej mierze pochodzą
	z~prezentacji pracownika Rackspace Sorena Hansena \cite{hansen11}.

	Projekt OpenStack narodził się w~czerwcu 2010 roku. Twórcami są
	dostarczająca usługi hostingowe firma Rackspace oraz amerykańska agencja
	NASA. Obie organizacje pracowały już wcześniej nad rozwiązaniami w~chmurze
	i~rozważały udostępnienie wyników swoich prac jako open source. Po
	opublikowaniu przez NASA projektu Nebula Rackspace zaproponował im
	współpracę mającą na celu utworzenie otwartej platformy cloud computing.

	W~lipcu 2010 w~Austin w~Teksasie zorganizowane zostało spotkanie OpenStack
	Design Summit. Wzięło w~nim udział około 200 osób, dużo więcej niż
	zakładali organizatorzy. Wkrótce później NASA, Rackspace i~kilkadziesiąt
	innych firm zadeklarowało chęć współpracy nad rozwojem projektu OpenStack.
	Pierwsza wersja projektu pod nazwą kodową Austin została opublikowana
	w~październiku 2010. Kod udostępniono na otwartej licencji Apache License.

	Zainteresowanie OpenStack rośnie bardzo szybko. Wykorzystaniem go
	zainteresował się m.in. rząd Kanady \cite{computerworld_bort}. Zaufanie
	wobec OpenStack buduje oferta dla użytkowników biznesowych. Poza
	standardową pomocą jakiej może udzielić użytkownikom społeczność, firma
	Rackspace oferuje wsparcie komercyjne dla klientów zainteresowanych
	uruchomieniem OpenStack na swojej architekturze
	\cite{openstack_goes_commercial}. Nie bez znaczenia dla estymy projektu jest
	też lista deklarujących zaangażowanie w~projekt firm, takich jak choćby
	Dell, Intel, AMD czy Cisco \cite{openstack}.

	\section{Polityka otwartości}

	Twórcy OpenStack od początku deklarowali chęć dążenia do otwartości. Na wiki
	projektu opublikowali swoją definicję tego pojęcia i~zobowiązali się do
	przestrzegania przestawionych tam zasad \cite{openness}. Zagwarantowali, że
	prace nad OpenStack będą prowadzone publicznie i~żadne decyzje w~sprawie
	projektu nie będą podejmowane bez uczestnictwa społeczności. Obiecali
	nie ograniczać funkcjonalności swojego produktu w~celu czerpania zysków
	z~płatnej, bogatszej wersji platformy.

	Repozytoria z~kodem źródłowym OpenStack są publicznie dostępne na serwisie
	Launchpad. Kod jest dostępny na licencji Apache License. Zgodnie
	z~obietnicami twórców prace nad projektem odbywają się publicznie przy
	pomocy dostępnych w~ramach serwisu Launchpad grup dyskusyjnych, systemów
	zarządzania dokumentacją i~bugami.

	Licencja Apache, na której dostępny jest kod OpenStack, jest otwartą
	licencją pozwalającą wszystkim licencjobiorcom na dowolne używanie
	opublikowanego na jej zasadach oprogramowania \cite{apache_license_faq}. Nie
	jest ona licencją jednak typu copyleft. Dzieła utworzone z~wykorzystaniem
	kodu udostępnionego na jej warunkach nie muszą być udostępniane tych samych
	zasadach. Pozwala to na utworzenie w~oparciu o~OpenStack komercyjnego
	produktu i~licencjonowanie go na dowolnie ustalonych przez siebie zasadach.

	Kontrowersje na temat zadeklarowanej otwartości wzbudziła w~marciu bieżącego
	roku decyzja Rackspace o~wprowadzeniu nowego modelu zarządzania projektem.
	Sama treść ich propozycji nie budziła sprzeciwów. Niepokojący jednak był
	fakt, że proces przygotowania nowego modelu nie był publiczny a~decyzji
	o~jego wprowadzeniu skonsultowano ze społecznością.

	Na powyższy fakt uwagę zwrócił jeden z~głównych autorów OpenStack, Rick
	Clark, w~swojej wiadomości na grupie dyskusyjnej projektu
	\cite{clark_list_msg}. Wyraził tam swoje zastrzeżenia do sposobu
	wprowadzenia zmiany. Wkrótce później podjął on decyzję o~rezygnacji ze
	stanowiska w~Rackspace a~szczegóły sprawy opisał na swoim blogu
	\cite{why_i_left_rackspace}. Zwrócił tam uwagę na fakt, że jego zdaniem
	Rackspace zamiast wpływać na kierunek rozwoju OpenStack chce go kontrolować.
	Zarzucił swoim byłym pracodawcom, że ich celem jest zbijanie na projekcie
	kapitału PR-owego zamiast doskonalenia technologii. Stwierdził również, że
	wkład większości oficjalnych partnerów Rackspace, którzy zadeklarowali
	współpracę przy projekcie, ograniczył się do notatki w~zakładkach dla
	mediów. Na koniec zasugerował, że kod i~marka OpenStack powinna należeć do
	niezależnej fundacji aby uniknąć w~przyszłości problemów podobnych jakie
	spotkały projekty firmy Sun po wykupieniu jej przez Oracle.

	Firma Rackspace ograniczyła swój komentarz na temat powyższego zajścia do
	stwierdzenia, że zmiany w~modelu zarządzania projektem zostały dobrze
	odebrane a~oni sami wyrażają zdecydowany entuzjazm w~związku z~przyszłością
	projektu \cite{concerns_about_openness}.

	\section{Konkurencja}

	Obecnie w świecie open source istnieje tylko jeden konkurencyjny wobec
	OpenStack projekt. Jest nim Eucalyptus, który narodził się jako projekt
	badawczy w~University of California niemal cztery lata temu, a~od roku 2009
	funkcjonuje jako produkt komercyjny \cite{openstack_vs_euca}. Grupa docelowa
	obu projektów jest podobna -- klienci życzący sobie uruchomienia w~swojej
	infrastrukturze prywatnych lub publicznie dostępnych środowisk typu cloud
	computing.

	Kluczową różnicą jest fakt, że Eucalyptus, w~przeciwieństwie do założeń
	przyjętych przez twórców OpenStack, oferuje swoim klientom komercyjną wersję
	Eucalyptus Enterprise Edition. Rozszerza ona zestaw funkcji m.in. o~lepsze
	wsparcie z~VMware i~niektórymi systemami baz danych
	\cite{openstack_vs_euca}, ale jej kod źródłowy nie jest otwarty.

	Autorzy Eucalyptus i~OpenStack twierdzą, że nie stoją do siebie w~opozycji
	i~widzą duże możliwości współpracy nad udoskonalaniem rozwiązań klasy cloud
	computing. Przeciwko tym oświadczeniom świadczy jednak fakt, że oba produkty
	mają podobną funkcjonalność. Dowodzi temu zastosowanie Eucalyptus w~NASA, do
	czasu aż ze względu na zamknięty charakter tego rozwiązania podjęto decyzję
	o~jego porzuceniu. Wkrótce potem NASA rozpoczęła współpracę z~Rackspace nad
	OpenStack.

	\paragraph{}

	Poza rozwiązaniami ze świata open source istnieją różne inne platformy IaaS,
	jak choćby Amazon Web Services i~produkty innych konkurujących z~Amazon
	usługodawców. Oferują one podobną do OpenStack funkcjonalność a~jakość ich
	usług została wyszlifowana na przestrzeni wielu lat istnienia na rynku.
	Pominięto je jednak w~tej analizie ze względu na odmienny, stricte
	komercyjny charakter.

	\section{Analiza repozytorium}

	Kod źródłowy projektów wchodzących w skład OpenStack jest dostępny
	w~repozytoriach Bazaar przechowywanych w~serwisie Launchpad. Zawartość
	repozytoriów Nova i~Swift -- odpowiednio systemu zarządzającego maszynami
	wirtualnymi oraz bazy danych typu klucz-wartość -- została pobrana i~poddana
	analizie.

	\begin{table}
		\centering
		\begin{tabular}{|r|r|r|}
			\hline
			\bf Projekt & \bf Komentarz & \phantom{asdf} \bf Kod \\
			\hline
			Nova & 12157 & 39854 \\
			Swift & 6090 & 32011 \\
			\hline
		\end{tabular}
		\caption{Wyniki analizy kodu OpenStack narzędziem CLOC}
		\label{tab:cloc}
	\end{table}

	Obydwa projekty są rozwijane w~języku Python. Przy pomocy narzędzia CLOC
	\cite{cloc} sprawdzono liczbę linii zawierających komentarze w~porównaniu do
	linii z~kodem źródłowym. Wyniki przedstawiono w~tabeli \ref{tab:cloc}.
	Na ich podstawie możemy stwierdzić, że OpenStack ze średnim udziałem
	komentarzy w~wysokości około 20\%, plasuje się poniżej średniej dla
	otwartych projektów w~języku Python, która według serwisu Ohloh wynosi 24\%
	\cite{ohloh_python_stats}. Biorąc jednak pod uwagę, że w~przypadku obydwu
	projektów w~ich katalogach \texttt{doc} znajduje się obszerna dokumentacja,
	możemy uznać, że autorzy projektów Nova i~Swift poważnie podchodzą do
	dokumentowania tworzonych przez siebie narzędzi.

	Z~historii repozytoriów uzyskano informacje o~adresach e-mail autorów
	dokonywanych zmian. W~przypadku obydwu projektów autorami największej liczby
	zmian byli pracownicy firmy Rackspace, w~wypadku Swift ponad 65\%,
	a~w~wypadku Nova ponad 30\%. Interesujący jest również fakt, że
	najaktywniejszy programista w~projekcie Nova wydaje się nie być pracownikiem
	Rackspace. Świadczy o~tym jego adres spoza domeny rackspace.com.
	W~przypadku projektu Nova unikalnych adresów e-mail było 98, w~przypadku
	Swift -- 29. Świadczy to o~fakcie, że autorom OpenStack w~krótkim
	czasie, jaki minął od jego powstania, udało się zainteresować swoim
	projektem relatywnie spore grono programistów.

	\section{Podsumowanie}

	OpenStack to projekt bardzo młody, ale jego autorzy dołożyli starań, aby od
	samego początku rozwijać go zgodnie z~najlepszymi wzorcami ze świata open
	source. Mnogość dróg komunikacji między użytkownikami a~twórcami,
	dostępność dokumentacji, publiczny system continuous integration oraz kod
	źródłowy w~rozproszonym systemie kontroli wersji to atuty, które bardzo
	dobrze wróżą przyszłości projektu.

	Z~drugiej jednak strony sytuacja opisana przez Ricka Clarka budzi pewne
	wątpliwości co do intencji obecnego właściciela nazwy OpenStack, firmy
	Rackspace. Warto śledzić rozwój wypadków aby przekonać się czy koordynatorzy
	projektu są tak szczerze przekonani o~wadze otwartości jak twierdzą.
	Licencja Apache, na której opublikowany jest projekt, pozwala autorom na
	rezygnację z~udostępniania kodu kolejnych wersji projektu, co warto wziąć
	pod uwagę rozważając wykorzystanie OpenStack w~swojej architekturze.

	\bibliographystyle{../plplainurl}
	\bibliography{../bibliografia}

\end{document}
