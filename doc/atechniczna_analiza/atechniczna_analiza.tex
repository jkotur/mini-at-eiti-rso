\documentclass[a4paper,12pt]{article}

\usepackage[utf8]{inputenc}
\usepackage{polski}
\usepackage{a4wide}

\usepackage{hyperref}
\hypersetup{
	unicode = true,
	pdfauthor = {Jan Stępień},
}

\title{Analiza atechniczna}
\author{Jan Stępień}

\hyphenation{IaaS}

\begin{document}
	\maketitle

	\section{Wstęp}

	Projekt OpenStack to zestaw narzędzi umożliwiających utworzenie platformy
	typu IaaS, będącej podstawową warstwą zyskującej popularność architektury
	w~chmurze. W~jego skład wchodzą m.in. komponenty ułatwiające nadzorowanie
	pracy maszyn wirtualnych oraz umożliwiające redundancyjne przechowywanie
	plików i~dostęp do nich. Projekt ma charakter open source a~jego kod jest
	publicznie dostępny.

	Celem niniejszego dokumentu jest przedstawienie wyników subiektywnej,
	nietechnicznej analizy projektu OpenStack. Motywacją do jej przygotowania
	jest przekonanie autora, że takie cechy jak bogata funkcjonalność,
	kompleksowy garnitur testów i~bogata dokumentacja nie są warunkami
	wystarczającymi do zaufania danemu rozwiązaniu. W~projektach o~otwartym
	kodzie źródłowym istotną rolę odgrywa wiele innych czynników, takich jak
	choćby przezroczysty proces wytwórczy, efektywność kanałów komunikacji
	między użytkownikami i~twórcami oraz aktywna społeczność wspierająca prace
	wytwórcze.

	\section{Historia}

	Projekt OpenStack powstał w~czerwcu 2010 roku \cite{hansen11}. Twórcami byli
	dostarczająca usługi hostingowe firma Rackspace oraz amerykańska agencja
	NASA.

	\bibliographystyle{../plplainurl}
	\bibliography{../bibliografia}

\end{document}
