\subsection{Swift}

Projekt OpenStack Swift jest jeszcze na dość wczesnym etapie rozwoju. W
efekcie czasem można natknąć się na ewidentne błędy autorów [\todo{tu wstawić opis
błędu}]. Z drugiej jednak strony, na korzyść tegoż projektu przemawia proste
API i dość dobra dokumentacja. Te dwie cechy sprawiają, że korzystanie ze
Swifta jest całkiem proste.

Przedstawione przez nas na początku semestru w~dokumencie \cite{jano-anal-atech}
obawy na temat otwartości procesu powstawania projektu Swift nie okazały się
przesadzone. Jak zauważyliśmy w~sekcji \ref{sec:zaleznosci}, niepokoi nas brak
odpowiedzi na zgłoszone przez nas błędy. Mamy nadzieję, że całość będzie
zmieniać się na lepsze, ale zespołom planującym wykorzystanie Swift w~swoich
rozwiązaniach zalecamy wzięcie tej kwestii pod uwagę.

\subsection{python-fuse}

Zdecydowaliśmy się na implementację systemu plików w Pythonie. Biblioteka FUSE
posiada interfejs w języku C, co implikowało konieczność użycia jakichś
bindingów dla Pythona. Wybór padł na python-fuse. Wygoda ich użycia była
niestety mocno ograniczona przez brak dokumentacji. Na szczęście w internecie
można znaleźć kompletną, "pustą", przykładową implementację. Na jej podstawie
udało nam się zorientować, jak mają wyglądać wszystkie potrzebne funkcje.

