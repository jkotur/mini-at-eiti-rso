	\subsection{Stan projektu}\label{sec:stan projektu}

	Projekt w momencie kiedy pisany jest ten dokument jest w pełni funkcjonalnym systemem plików. Można wykonywać takie operacje jak tworzenie i usuwanie katalogów i plików, zapisywanie danych do pliku, oraz tworzenie linków symbolicznych. Dokładny opis tego co zostało zrobione i jak działa, zawarty jest w dokumentacji projektu. Pierwotne założenia projektu zakładały też pełny system ACL dla plików i katalogów. Po kilku próbach rozwiązania tego problemu, doszliśmy do wniosku, że nie ma możliwości prostego sposobu na implementacje tej funkcjonalności. Ten dokument opisze nasze prace nad tym zagadnieniem i wnioski jakie z nich wyciągnęliśmy.
	
	\subsection{Wsparcie ACL przez Swift}\label{sec:wsparcie acl przez swift}

	Pierwsze podejście do zagadnienia systemu autoryzacji w Swifcie było bardzo pozytywne. Według dokumentacji Swift wspiera pełny system ACL, więc sądziliśmy że nie będzie najmniejszego problemu z użyciem go w naszym projekcie. Aby użyć tego systemu, trzeba używać serwera autoryzacji swauth, ale tutaj też czekała nas miła niespodzianka, ponieważ w wersji 1.3 Swifta, domyślnym systemem autoryzacji jest swauth. Szybko udało się więc skonfigurować wszystko, tak aby utworzyć konta użytkowników i zacząć testy.

	W dokumentacji jest napisane, że aby ustawić uprawnienia użytkownikowi do kontenera, należy wpisać w dodatkowych nagłówkach \texttt{'x-container-read'} i \texttt{'x-container-write'} użytkownika. Po kilku próbach okazało się, że poprawną nazwą użytkownika jest \texttt{<nazwa konta>:<login użytkownika>}. Po wpisaniu przez administratora poprawnych wartości w~powyższe pola użytkownik uzyskuje dostęp do kontenera.

	Niestety na tym skończyły się sukcesy z używaniem systemu ACL. Okazało się że użytkownik nie jest w stanie przeczytać zawartości konta, do którego się zalogował. Jedyne co jest napisane w dokumentacji na temat autoryzacji, to wymienione powyżej nagłówki. Na próżno szukaliśmy kolejnych informacji na temat autoryzacji w Swifcie w Internecie. Dopiero zajrzenie w~źródła projektu dało odpowiedź na pytanie, co dalej. Okazało się że jedynym miejscem gdzie Swift komunikuje się z swauth i wymienia informacje o ACLkach są właśnie kontenery. Działa to w ten sposób, że jeżeli użytkownik ma uprawnienia do pisania do kontenera, to może tworzyć i nadpisywać wszystkie obiekty w nim zawarte, analogicznie jest z czytaniem. Oznacza to też że nie ma żadnej możliwości operacji na kontenerach przez użytkownika. Nie może on ani listować, ani tworzyć kontenerów.

	Pomyśleliśmy więc że musimy sobie radzić z tym co mamy i rozpatrzyliśmy dwie zupełnie odmienne architektury.
	
	\subsection{Projekty architektury}\label{sec:projekty architektury}

	\subsubsection{Projekt bezpośredni}\label{sub:projekt bezposredni}

	Pierwszym projektem kiedy jeszcze nie wiedzieliśmy jak wygląda system autoryzacji w Swifcie, był projekt bezpośredni, gdzie folderowi odpowiada kontener, a plikowi obiekt. Jest to prosty i intuicyjny projekt, w którym synchronizacja i tranzakcyjność zapewnione są przez Swifta. Dokładny opis tego rozwiązania znajduje się w dokumentacji, ponieważ ostatecznie zdecydowaliśmy się na tę wersję architektury. W tym miejscu należy zauważyć, że pierwotnie pomysł ten przepadł ze względu na kompletny brak możliwości implementacji uprawnień przy pomocy Swifta. Ponieważ wszystkie obiekty mają takie same uprawnienia w danym kontenerze, a kontenerów w koncie nie da się w ogóle listować, taki projekt nie mógł mieć systemu uprawnień zapewnionego przez Swifta.
	
	\subsubsection{Projekt kontenerów}\label{sub:projekt kontenerow}

	Drugim projektem, który miał duże szanse powodzenia, był pomysł tworzenia dla każdego obiektu (czyli katalogu, pliku, dowiązania) kontenera opisującego ten obiekt. Pomysł jest już trochę trudniejszy w implementacji, ponieważ należy dbać o synchronizację i poprawność struktury kontenerów. Może też cierpieć na utratę danych, ponieważ jeżeli utracimy opis katalogu, tracimy też dostęp do wszystkich podkatalogów. W zamian zyskujemy o wiele szybsze operacje przenoszenia i zapisu danych. Z punktu widzenia systemu autoryzacji ten projekt też był bardzo obiecujący. Ponieważ w Swifcie tyko kontenery mają swoje uprawnienia, to bylibyśmy w stanie opisać uprawnienia systemu plików, uprawnieniami Swifta. Istnieje jednak jeden istotny problem, który przekreślił to rozwiązanie zupełnie. Okazało się, że zwykły użytkownik w systemie Swift nie jest w stanie utworzyć kontenera. Ten fakt spowodował, że również to rozwiązanie zostało porzucone.

	\subsubsection{Podsumowanie}\label{sub:podsumowanie}
	
	Oba rozwiązania przedstawione w tym rozdziale są poprawne i dobrze spełniają swoje zadanie. W obu jednak nie byliśmy w stanie zapewnić autoryzacji przy pomocy narzędzi udostępnionych przez Swifta. Podejrzewamy że spowodowane to jest młodym stadium rozwoju projektu Swift. Ponieważ Swift jest bardzo aktywnie rozwijany, najprawdopodobniej niedługo pojawią się narzędzia pozwalające na szersze nadawanie uprawnień, tak że będzie można z poziomu użytkownika tworzyć katalogi. Najlepszym rozwiązaniem jest więc poczekać i zobaczyć które rozwiązanie szybciej stanie się możliwe.
	
	\subsection{Możliwe rozwiązania}\label{sec:mozliwe rozwiazania}

	Poniżej przedstawione zostaną rozwiązania które są możliwe do zaimplementowania ze Swiftem w aktualnej wersji, jednak z różnych powodów nie widzieliśmy sensu ich implementacji. Z powodu ograniczeń w Swifcie, są one kombinowane, także zdecydowanie lepszym rozwiązaniem jest poczekać.

	\subsubsection{Lokalna autoryzacja}\label{sub:lokalna autoryzacja}

	Aby zapewnić wrażenie autoryzacji i systemu ACL, można stworzyć system autoryzacji lokalnie, tak aby aplikacja sama weryfikowała uprawnienia. To rozwiązanie ma jednak jedną zdecydowaną wadę: zabezpieczenie jest zupełnie fikcyjne. Pomijając ataki typu crackerskiego, jak podmienianie kodu aplikacji, każdy użytkownik znając swoje hasło i swój login, mógłby się po prostu zalogować do Swifta korzystając z jego API i przeczytać pliki mu niedostępne. Szczególnie złośliwy włamywać mógłby też podmienić pliki używane przez nas do autoryzacji, tak aby inni użytkownicy nie mieli do nich dostępu. Takie rozwiązanie, choć możliwe, nie zapewnia żadnego stopnia bezpieczeństwa, co spowodowało, że nie zdecydowaliśmy się na jego implementację.

	\subsubsection{Rozszerzenie projektu Swift}\label{sub:rozszerzenie projektu swift}

	Ten sam kod, który jest potrzebny do lokalnej autoryzacji, wykonywany zdalnie na serwerze ma już sens. Można zatem uruchomić serwer pośredniczący zajmujący się wyłącznie autoryzacją, albo po prostu rozszerzyć o~tę funkcjonalność projekt Swift. Wystarczyło by zapewne dodać obsługę pól nagłówków \texttt{'x-account-*'} oraz \texttt{'x-object-*'}, analogiczną jak w przypadku \texttt{'x-container-*'}. To rozwiązanie, jakkolwiek dobre, wykracza poza zakres projektu zdalnego systemu plików.
	
	\subsubsection{Wykorzystanie projektu Nova}\label{sub:wykorzystanie projektu nova}

    Rozważaliśmy także możliwość wprowadzenia warstwy pośredniczącej między samym Swiftem a~naszym systemem plików. Warstwa ta odpowiadałaby za autoryzację i~posiadała pełen dostęp do chmury Swifta. Warstwę taką można by zrealizować z~wykorzystaniem projektu Nova. Należałoby umieścić w chmurze kilka wirtualnych serwerów, które sprawdzałyby, do jakich zasobów użytkownik ma dostęp. Klient łączyłby się do serwera, a~ten przekazywałby mu dane, po uprzedniej autoryzacji użytkownika i sprawdzeniu jego uprawnień. Sama informacja o uprawnieniach mogłaby być wówczas zakodowana wewnątrz obiektów Swifta.
	
	\subsection{Wnioski}\label{sec:wnioski}

	Reasumując, stworzyliśmy dwa projekty zdalnego systemu plików, z których każdy ma swoje wady i zalety. Obu jednak nie jesteśmy w stanie stworzyć przy aktualnym stanie projektu Swift. Mimo że Swift wspiera system ACL, jest on zbyt ubogi, aby stworzyć system plików z autoryzacją po stronie serwera. Użytkownik ma albo zbyt małe, albo zbyt duże uprawnienia, do poszczególnych obiektów Swifta. Wpadliśmy też na kilka pomysłów jak zaimplementować system autoryzacji korzystając z aktualnej wersji Swifta, jednak żaden z nich nie był dość zadowalający. Dobre rozwiązanie są zbyt trudne, a te dość proste, są z gruntu błędne. Najlepszym więc rozwiązaniem według nas jest poczekać na rozwój projektu Swift i w zależności od tego jak się rozwinie, zaimplementować jedno z dobrych rozwiązań.
