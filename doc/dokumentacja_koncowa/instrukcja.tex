Użycie naszego projektu różni się od standardowego użycia systemu plików
jedynie sposobem montowania i~demontowania. Cała reszta poleceń powłoki
i~wywołań I/O~jest identyczna.

\subsection{Zależności}\label{sec:zaleznosci}

Do instalacji i~poprawnego działania projektu \cb{} niezbędna jest instalacja
następujących aplikacji i~biblioteki, wraz z~ich zależnościami.

\begin{itemize}
	\item{Python 2.6}
	\item{python-fuse 0.2.1}
	\item{swift, nowszy lub równy wersji 1.3}
\end{itemize}

Biblioteka kliencka Swifta jest jeszcze mocno niedopracowana. Do tego stopnia,
że podstawowe wywołania API rzucają wyjątkami wskazującymi na literówki
w~kodzie Swift. Od wersji 1.3 wzwyż, tylko jeden z~tego typu wyjątków był
spowodowany błędami w~naszym kodzie. Aby uruchomić projekt, należy usunąć 3
linie odpowiedzialne za logowanie w~kodzie Swifta -- to one powodują błędy
i~wyjątki. Warto w~tym miejscu zaznaczyć, że na początku semestru zgłosiliśmy
ten błąd developerom Swifta i~do tej pory nie uzyskaliśmy
odpowiedzi~\cite{launchpad-bug}.

\subsection{Instalacja}

Dalsze sekcje będą zakładały, że została wykonana poprawna instalacja a~główny
skrypt programu znajduje się w~jednym z~katalogów ze~zmiennej środowiskowej
\texttt{PATH}. Poniższe polecenie wykonane w~głównym katalogu ze źródłami
zainstaluje \cb{} w~katalogu \texttt{/usr/local}.

\begin{verbatim}
./setup.py install
\end{verbatim}

Jeżeli \cb{} ma być zainstalowany w~innym miejscu należy wykorzystać opcję
\texttt{--prefix}:

\begin{verbatim}
./setup.py install --prefix /absolute/destination/path
\end{verbatim}

Więcej informacji na temat procedury instalacji i~możliwych opcji można uzyskać
przy pomocy opcji \texttt{--help}.

\begin{verbatim}
./setup.py --help
\end{verbatim}

\subsection{Montowanie}

Aby zamontować system plików używamy polecenia:

\begin{verbatim}
cumulonimbus <mountpoint> <opcje>
\end{verbatim}

Opcje wbrew nazwie nie są opcjonalne. Pierwsza z~nich to authurl -- adres URL
serwera autoryzacji. Kolejne dwie to user oraz key -- czyli użytkownik oraz
jego hasło na tym serwerze.

Opcje podajemy po fladze -o:

\begin{verbatim}
cumulonimbus dir -o authurl=http://127.0.0.1:8080/auth/v1.0 \
    -o user=test:tester -o key=testing
\end{verbatim}

Warto zwrócić uwagę na charakterystyczny dwukropek w~nazwie użytkownika.
Oddziela on pierwszy człon nazwy -- konto -- od drugiego -- faktycznej nazwy.
Na jednym koncie może znajdować się wielu użytkowników, każdy z~innymi
uprawnieniami.

\subsection{Demontowanie}

Demontowanie jest na szczęście znacznie prostsze:

\begin{verbatim}
fusermount -u <mountpoint>
\end{verbatim}
