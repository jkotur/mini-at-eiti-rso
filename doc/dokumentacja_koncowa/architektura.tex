\subsection{Swift}

Do realizacji projektu \cb{} wykorzystany został projekt OpenStack Storage, znany
również jako Swift. Wyczerpujący opis tego narzędzia znajduje się w~jednym
z~dokumentów dostarczonych na początku semestru \cite{qba-swift}, stąd
w~niniejszym dokumencie nie będziemy zajmować się szczegółami jego działania.
Ograniczamy się jedynie do przypomnienia, że Swift to rozproszony system
bazodanowy pozwalający na redundancyjne przechowywanie plików w~kontenerach
o~płaskiej strukturze.

\subsubsection{Konfiguracja}

Projekt \cb{} wymaga do swojego działania dostępu do działającego klastra Swift
o~dowolnej architekturze. Prace deweloperskie były prowadzone na minimalnej
instancji uruchomionej na pojedynczej maszynie wirtualnej. Testy w~laboratorium
wykonywano na bardziej rozbudowanej instalacji z~serwerami danych rozproszonymi
na kilku maszynach fizycznych.

Sposób konfiguracji minimalnej instancji Swift potrzebnej do działania projektu
\cb{} -- Swift All In One -- został przedstawiony w~dokumentacji Swift
\cite{swift_doc}. Rozwiązanie to powoduje uruchomienie serwera autoryzującego,
serwera pośredniczącego oraz czterech serwerów danych na jednej maszynie
fizycznej. Alternatywne rozwiązania umożliwiają uruchomienie każdego
z~powyższych serwerów na różnych maszynach fizycznych, co według zapewnień
autorów Swift powinno nadawać całemu rozwiązaniu cechy systemów
\textit{fault-tolerant}.

\subsubsection{Zalety}

Projekt \cb{} jest niezależny od sposobu konfiguracji klastra Swift. Może być to
zarówno minimalna instancja uruchomiona na pojedynczej maszynie fizycznej, jak
i~rozproszony system wielu instancji. Dzięki poziomowi przezroczystości
zapewnianej przez Swift sposób jego instalacji nie ma znaczenia dla projektu
\cb{}. Przezroczystość ta pozwala projektowi \cb{} na udostępnienie
użytkownikowi systemu plików o~szeregu zalet.

Po pierwsze, zgodnie z~zapewnieniami autorów, Swift gwarantuje redundancyjne
mechanizmy przechowywania danych oraz \textit{fault-tolerance}. W~wypadku
uszkodzenia pewnej części węzłów klastra przechowujących dane system plików
\cb{} może nadal działać poprawnie, pod warunkiem, że awaria nie zaburzy
działania samego systemu Swift. Jeżeli zapewnienia autorów Swift są poprawne,
użytkownik nie powinien odczuć żadnych konsekwencji tego typu zdarzeń.

Drugą zaletą jest możliwość złączenia w~ramach klastra Swift dowolnie dużej
przestrzeni dyskowej. W~efekcie \cb{} pozwala na przechowywanie wielkiej ilości
danych, które zostaną automatycznie rozlokowane pomiędzy dyskami wchodzącymi
w~skład maszyn tworzących wykorzystywany przez \cb{} klaster.

Trzecią zaletą jest wysoki poziom abstrakcji interfejsu Swift, który udostępnia
jedynie podstawowe funkcje odpowiadające za mechanizmy przechowywania
i~odzyskiwania danych. Mechanizmy redundancyjnego rozmieszczania danych
w~klastrze, wydajnego wyszukiwania obiektów potrzebnych do zrealizowania żądania
użytkownika czy też zapewnienie spójności pomiędzy serwerami danych jest
dokonywane automatycznie. Jako autorzy systemu plików nie musimy też rozwiązywać
skomplikowanego zagadnienia jakim jest fragmentacja partycji.

\subsection{FUSE}

\begin{figure}
	\centering
	\includegraphics[width=0.6\textwidth]{fuse.pdf}
	\caption{Architektura systemu}
	\label{fig:fuse}
\end{figure}

W~systemie operacyjnym Linux systemy plików implementowane są i~działają w
przestrzeni adresowej jądra. Istnieje jednak możliwość implementacji takiego
systemu w~przestrzeni użytkownika. Pozwala na to biblioteka FUSE -- Filesystem
in Userspace. Odpowiada ona za przekazywanie niskopoziomowych wywołań systemu
plików do procesu użytkownika.

Dzięki temu możliwa jest implementacja pełnoprawnego systemu plików bez
jakiejkolwiek ingerencji w~kod jądra. Niewątpliwą wygodą tego rozwiązania
jest fakt, że żaden błąd w~takiej implementacji nie może zakłócić działania
jądra systemu -- inaczej niż w~tradycyjnych systemach plików.

Kolejną zaletą tej biblioteki jest obecność bindingów do różnych języków
programowania, w~tym do Pythona, na którego użycie się zdecydowaliśmy. Diagram
prezentujący przepływ żądania operacji na zamontowanej partycji znajduje się na
rysunku \ref{fig:fuse}.

\subsection{Implementacja systemu plików}

Projekt \cb{} implementuje prosty system plików. Wspiera kluczowe dla normalnej
pracy operacje, takie jak tworzenie, odczyt, modyfikacja i~usuwanie katalogów
i~plików. Wspierane są również dowiązania symboliczne. Poza zakresem prac
znalazły się wielokrotne dowiązania twarde -- każdy plik ma tylko jeden hard
link.

Każdy katalog jest reprezentowany jako osobny kontener o~unikalnej nazwie będącą
ścieżką danego katalogu zakodowaną przy pomocy kodowania \textit{URL-encoding}.
Wewnątrz kontenera przechowywane są znajdujące się w~odpowiadającym mu katalogu
pliki i~dowiązania symboliczne. Każdy plik i~dowiązanie przechowywany jest jako
obiekt w~rozumieniu pakietu Swift. Nazwy obiektów to ostatnia część nazwy
odpowiedniego pliku lub dowiązania, również zakodowana przy pomocy
\textit{URL-encoding}. Do realizacji kodowania i~dekodowania
\textit{URL-encoding} wykorzystany został moduł \texttt{urlib} wchodzący w skład
standardowej biblioteki Pythona 2.6 i~nowszych.

Zawartość plików przechowywana jest jako zawartość reprezentujących je obiektów.
Dowiązania symboliczne są wyróżnione wartością przypisanego każdemu Swiftowemu
obiektowi nagłówka \texttt{content-type} -- w~ich wypadku ma ono wartość
\texttt{symlink}. Zwykłe pliki mają w~tym miejscu standardowy typ zawartości,
\texttt{application/octet-stream}. Ponadto treść obiektu reprezentującego
dowiązanie symboliczne to ścieżka do wskazywanego przezeń pliku. Informacje
o~czasie ostatniej modyfikacji obiektów przechowywane są w~analogicznym do
\texttt{content-type} nagłówku \texttt{last-modified}.

Przykładową reprezentację minimalnego systemu plików przedstawiono na rysunku
nr~\ref{fig:drzewo}. Z~lewej strony widoczne jest drzewo katalogów, z~prawej
natomiast trzy kontenery Swift przechowujące reprezentację drzewa. Na diagramie
pominięto nagłówek \texttt{content-type} zwykłych plików.

\begin{figure}
    \centering
    \includegraphics[width=0.8\textwidth]{drzewo.pdf}
    \caption{Przykładowe drzewo katalogów}
	\label{fig:drzewo}
\end{figure}

\paragraph{}
Opisany sposób implementacji posiada niestety pewne wady. Wynikają one z
przyjętej przez nas architektury oraz założeń Swifta. Przede wszystkim \cb{}
kiepsko radzi sobie z~dużymi plikami -- API Swifta nie pozwala na podmienianie
części pliku, więc każda zmiana w~pliku niesie ze sobą konieczność przesłania
całej jego zawartości.

Nie ma także możliwości zmiany nazwy obiektu w~kontenerze, przez co każda zmiana
nazwy pliku sprowadza się do pobrania go w~całości na lokalny komputer, po czym
wysłania z~powrotem z~inną nazwą. To samo dotyczy katalogów.

Ostatni problem, którego nie rozwiązaliśmy ze względu na założenie minimalności
implementacji, to synchronizacja podczas jednoczesnej pracy kilku użytkowników.
W~obecnej postaci \cb{} nadaje się tylko do jednoczesnego korzystania przez
jednego użytkownika.

\subsection{Możliwości rozwoju}

Komfort używania opracowanego systemu plików wzrósłby gdyby tryb jego działania
zamieniono na asynchroniczny. Obecnie sterowanie jest zwracane do procesu
wykonującego operację I/O dopiero po zakończeniu wykonywania zapytania HTTP do
Swift. Udoskonalenie polegałoby na realizacji żądań zapisu w~tle
i~natychmiastowym zwracaniu sterowania. Podobnie zrealizowane są operacje na
wolnych urządzeniach blokowych, np. pendrive'ach, w~systemie Linux. Poprawne
wykonanie operacji zapisu jest zagwarantowane dopiero po poprawnym odmontowaniu
urządzenia.

Realizacja tego pomysłu wiązała by się jednak z~wprowadzeniem w~projekcie
wielowątkowości lub wieloprocesowości, co istotnie skomplikowałoby jego
architekturę. Niezbędne byłyby mechanizmy synchronizacji weryfikujące przed
odmontowaniem czy wszystkie zlecenia zapisu zakończyły się.

\paragraph{}

Kolejna możliwość rozwoju projektu mogłaby polegać na uniezależnieniu go od
Swift i~umożliwieniu rozszerzania go o~wtyczki pozwalające na współpracę
z~innymi usługami przechowywania danych w~chmurze. Przykładowymi backendami
wspieranymi poprzez tego typu wtyczki byłyby stanowiąca inspirację dla Swift
usługa Amazon Simple Storage Service lub popularny serwis Dropbox.
Funkcjonalność tego typu uczyniłaby projekt odpornym na ewentualne zmiany na
rynku usług przechowywania danych w~chmurze, jak choćby pojawienie się nowych,
lepszych rozwiązań lub zawieszenie prac nad projektem Swift.

\paragraph{}

Aby \cb{} mógł stać się rozwiązaniem, które sprawdzi się w~praktyce, konieczna
byłaby implementacja mechanizmu synchronizacji, który zapobiegłby uszkodzeniu
plików, na przykład poprzez uniemożliwienie jednoczesnej edycji przez różnych
użytkowników. Mechanizm taki musiałby radzić sobie z~sytuacjami, w~których
zerwaniu ulega połączenie z~użytkownikiem blokującym dostęp do pliku innym.

\paragraph{}

Wprowadzenie obługi uprawnień użytkowników byłoby kolejnym krokiem w~stronę
stworzenia systemu, który pozwalałby na obsługę wielu użytkowników. Oczywiście
wszelkie uprawnienia muszą być wspierane przez samą chmurę -- w~tym przypadku
Swifta. W~przeciwnym przypadku, możnaby je łatwo ominąć, nie korzystając
z~projektu \cb{}, lecz z~innego połączenia z~cloudem.
