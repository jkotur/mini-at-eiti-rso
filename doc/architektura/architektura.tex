\documentclass[a4paper,12pt]{article}

\usepackage[utf8]{inputenc}
\usepackage[polish]{babel}
\usepackage{polski}
\usepackage{a4wide}

\usepackage[colorlinks=true,linkcolor=blue]{hyperref}

\usepackage[pdftex]{graphicx}



\hypersetup{
	unicode = true,
	pdfauthor = {Jan Stępień},
}

\title{Cumulonimbus -- Architektura Systemu}
\author{Jan Stępień}

\def\cb{Cumulonimbus}
\def\todo{\textbf{TODO}: }

\begin{document}

\maketitle

\section{Swift}

Do realizacji projektu \cb{} wykorzystany został projekt OpenStack Storage, znany
również jako Swift. Wyczerpujący opis tego narzędzia znajduje się w~jednym
z~dokumentów dostarczonych na początku semestru \cite{qba-swift}, stąd
w~niniejszym dokumencie nie będziemy zajmować się szczegółami jego działania.
Ograniczamy się jedynie do przypomnienia, że Swift to rozproszony system
bazodanowy pozwalający na redundancyjne przechowywanie plików w~kontenerach
o~płaskiej strukturze.

\subsection{Konfiguracja}

Projekt \cb{} wymaga do swojego działania dostępu do działającego klastra Swift
o~dowolnej architekturze. Prace deweloperskie były prowadzone na minimalnej
instancji uruchomionej na pojedynczej maszynie wirtualnej. Testy w~laboratorium
wykonywano na bardziej rozbudowanej instalacji z~serwerami danych rozproszonymi
na kilku maszynach fizycznych. \todo oby to była prawda.

\subsection{Zalety}

Nie ma on żadnych wymagań wobec sposobu konfiguracji klastra. Może być to
zarówno minimalna instancja uruchomiona na pojedynczej maszynie fizycznej, jak
i~rozproszony system wielu instancji. Dzięki poziomowi przezroczystości
zapewnianej przez Swift sposób jego konfiguracji nie ma znaczenia dla projektu
\cb{}.

\subsection{Ograniczenia}

\bibliographystyle{../plplainurl}
\bibliography{../bibliografia}

\end{document}
